\documentclass[11pt,a4paper]{article}

\usepackage[utf8]{inputenc}
\usepackage{parskip}
\usepackage{tabularx}
\usepackage{amsmath}
\usepackage{amssymb}
\usepackage{dsfont}
\usepackage{geometry}
\usepackage{booktabs}
\usepackage{centernot}
\usepackage{hyperref}
\geometry{a4paper, left=20mm, right=20mm, top=20mm, bottom=20mm}

\usepackage{fancyhdr}
\pagestyle{fancy}
\lhead{Anthony Catterwell}
\chead{\textsc{University of Edinburgh}}
\rhead{Stochastic Modelling}

\title{Stochastic Modelling Notes}
\author{Anthony Catterwell}

\begin{document}
\maketitle
\tableofcontents

\break{}

\section{Preliminaries}

\subsection{Conditional Probability}

\begin{itemize}

    \item \textbf{Definition 1.1.2} \emph{Conditional probability}
        \[
            P(A|B) \equiv \frac{P(A \cap B)}{P(B)}
        \]

    \item \textbf{Theorem 1.1.4} \emph{Law of Total Probability}
        \[
            P(A) = \sum_{i=1}^n P(A|B_i)P(B_i)
        \]
    \item \textbf{Theorem 1.1.8}
        \begin{align*}{}
            p(x) &= \sum_y p(x|Y=y)p(y)            & X, Y \text{discrete} \\
            p(x) &= \int   p(x|Y=y)f(y)\mathrm{d}y & X \text{discrete}, Y \text{continuous} \\
            f(x) &= \sum_y f(x|Y=y)p(y)            & X\text{continuous}, Y \text{discrete} \\
            f(x) &= \int   f(x|Y=y)f(y)\mathrm{d}y & X, Y \text{continuous} \\
        \end{align*}
\end{itemize}

\subsection{Conditional Expectation}

\begin{align*}{}
    E(X|Y = y) &= \sum_{x \in S} xp(x|Y=y) &\text{if $X$ is discrete} \\
    E(X|Y = y) &= \int_{-\infty}^{\infty} xf(x|Y=y)\mathrm{d}x &\text{if $X$ is continuous}
\end{align*}

\begin{itemize}

    \item \textbf{Theorem 1.2.3} \emph{Tower Property}\\
        For $X$ and $Y$ random variables
        \[
            E[E(X|Y)] = E(X)
        \]
        or in detail:
        \[
            E[E(X|Y)] =
            \begin{cases}{}
                \sum_{y \in S} E(X|Y = y) p_Y(y) & \text{if} \ Y \ \text{is discrete} \\
                \int_{-\infty}^\infty E(X|Y = Y)f_Y(y) \mathrm{d}y & \text{if} \ Y \
                \text{is continuous} \\
            \end{cases}
        \]

\end{itemize}

\subsection{Stochastic Processes}

\begin{itemize}

    \item \textbf{Definition 1.3.1} \emph{Stochastic process} \\
        A \emph{stochastic process} ${(X_t)}_{t \in T}$
        is an indexed collection of random variables.
        Set $T$ is called the \emph{index set}.
        The set $S$ of all possible states is referred to as the \emph{state space}
        of the process.

\end{itemize}

\section{Discrete Time Markov Chains}

\subsection{Basic Definitions}

\begin{itemize}

    \item \textbf{Definition 2.1.1} \emph{Markov property}\\
        A stochastic process is said to have the \emph{Markov property} if,
        given the present state, the future events are independent of the past.
        For discrete-time discrete-space processes ${(X_n)}_{n \in N}$ this
        property can be stated as
        \[
            P(X_{n+1} = j | X_n = i, X_{n-1} = i_{n-1},
            \ldots, X_0 = i_0) = P(X_{n+1} = j | X_n = i)
        \]
        for all $j, i, i_{n-1}, \ldots, i_0 \in S$ and $n \in \mathbb{N}$,
        and we also define
        $p_{ij}(n) \equiv P(X_{n+1} = j | X_n = i)$
        and refer to it as the (one step) transition probability from $i$ to $j$ at time $n$.

\end{itemize}

\subsection{Modelling examples}

\subsection{More complicated examples }

\subsection{Chapman-Kolmogorov equations}

\begin{itemize}

    \item \textbf{Notation} \\
        $p_{ij}^{(n)}$ denotes the probability of reaching state $j$ from state $i$ in $n$ periods.
    \item \textbf{Theorem 2.4.1} \emph{Chapman-Kolmogorov equations} \\
        \[
            p_{ij}^{n+m} = \sum_{k\in S} {p_{ik}^n} {P_{kj}^m}
        \]
        also
        \[
            p^{n+m} = {p^n}{p^m}
        \]
    \item \textbf{Corollary 2.4.2} \\
        If $P^n$ is defined to be the $n$th power of a matrix $P$, then
        \[ P^{(n)} = P^n \]
    \item \textbf{Theorem 2.4.7} \\
        A one-step transition matrix $P$ and the initial distribution $a^{(0)}$ completely
        characterises the DTMC, that is, all finite-dimensional probabilities can be calculated.

\end{itemize}

\subsection{Classification of states}

\begin{itemize}

    \item \textbf{Definition 2.5.1} \emph{Accessibility} \\
        A state $j$ is said to be accessible from state $i$, denoted $i \rightarrow j$, if
        $\exists n \geq 0 \backepsilon p_{ij}^{(n)} > 0$
    \item \textbf{Theorem 2.5.3} \emph{Communication is an equivalence relation}
        that is
        \begin{enumerate}
            \item $i \leftrightarrow i \quad \forall i \in S$ (reflexive)
            \item $i \leftrightarrow j \implies j \leftrightarrow i$ (symmetric)
            \item $i \leftrightarrow j, j \leftrightarrow i \implies i \leftrightarrow k$
                (transitive)
        \end{enumerate}
    \item \textbf{Definition 2.5.4} \emph{Communicating class} \\
        Let $C \subseteq S$. $C$ is a \emph{communicating class} if
        \begin{enumerate}
            \item $i \in C,\ j \in C \implies i \leftrightarrow j$
            \item $i \in C,\ i \leftrightarrow j \implies j \in C$
        \end{enumerate}
        If, in addition to these properties, we cannot leave $C$, that is
        \[
            \forall i \in C,\ \forall k \notin C \backepsilon i \centernot\rightarrow k
            \implies C \text{ is a closed, communicating class}
        \]
    \item \textbf{Definition 2.5.5} \emph{Irreducibility} \\
        A DTMC is \emph{irreducible} if the state space $S$ is a single (closed)
        communicating class, and it is called \emph{reducible} if it is composed of several
        communicating classes

\end{itemize}

\subsection{Transience and recurrence}

\begin{itemize}

    \item \textbf{Notation} \\
        \begin{align*}{}
            T_j          & = \text{min}\{n \geq 1 : X_n = j\} \\
            \varrho_{ij} & = P(T_j < \infty | X_0 = i)
        \end{align*}
        Note that for $i \neq j,\ \varrho_{ij} > 0 \iff i \rightarrow j$
    \item \textbf{Definition 2.6.1} \emph{Recurrence and transience} \\
        State $i$ is \emph{recurrent} if $\varrho_{ii} = 1$, and \emph{transient} if $\varrho < 1$
    \item \textbf{Lemma 2.6.2}
        \begin{align*}{}
            P(N_i = \infty | X_0 = i) &=
            \begin{cases}{}
                1 & \text{if $i$ is recurrent} \\
                0 & \text{if $i$ is transient} \\
            \end{cases} \\
            E(N_i | X_0 = i) &=
            \begin{cases}{}
                \infty & \text{if $i$ is recurrent} \\
                \frac{1}{1 - \varrho_{ii}} & \text{if $i$ is transient} \\
            \end{cases}
        \end{align*}
    \item \textbf{Theorem 2.6.3} \\
        If $i \rightarrow j$ but $\varrho_{ji} < 1$ then $i$ is transient.
    \item \textbf{Corollary 2.6.4}
        \begin{enumerate}
            \item If $i \rightarrow j$ and $i$ is recurrent then $\varrho_{ji} = 1$
            \item If $i \rightarrow j$ and $i$ is recurrent then $j$ is also recurrent
            \item If $i \rightarrow j$ and $j$ is transient then $j$ is also transient
            \item Recurrence and transience are class properties
        \end{enumerate}
    \item \textbf{Theorem 2.6.7} \\
        State $i$ is recurrent if and only if $\sum_{n=0}^{(n)} = \infty$
    \item \textbf{Lemma 2.6.9} \\
        $\sum_{k=0}^{\infty}a_k$ converges if $\lim_{k \to \infty} \frac{a_{k+1}}{a_k} < 1$,
        and it diverges if $\lim_{k \to \infty} \frac{a_{k+1}}{a_k} > 1$

\end{itemize}

\subsection{Positive and null recurrence}

\begin{itemize}

    \item \textbf{Theorem 2.7.1} A recurrent state $i$ is \emph{positive recurrent} if and only if
        \[
            p_{ii}^* = \frac{1}{m_{ii}} > 0
        \]
        where
        \[
            m_{ij} = E(T_j | X_0 = i) \quad \text{and} \quad p_{ij}^* = \lim_{N \to \infty}
            \frac{1}{N+1} \sum_{n=0}^N p_{ij}^{(n)}
        \]
        $p_{ij}^*$ is the mean proportion of time spent at $j$ when starting from $i$.

    \item \textbf{Lemma 2.7.2} \\
        \[
            \lim_{N \to \infty} \frac{1}{N+1} \sum_{n=0}^N f_n = \lim_{N\to \infty}
            \frac{1}{N+1} \sum_{n=0}^N f_{n+m}
        \]
        for any sequence ${(f_n)}_{n \in \mathbb{N}}$ which is bounded $|f_n| \leq b$ for all $n$,
        and for any $m \in \mathbb{N}$

        % $$\forall (f_n)_{n \in \mathbb{N}} \text{ sequence which is bounded } |f_n| \leq b \quad \forall n, m \in \mathbb{N}$$

    \item \textbf{Theorem 2.7.3}
        Positive recurrence and null recurrence are class properties, that is if recurrent states
        $i \leftrightarrow j$ then $i$ and $j$ are both positive or null recurrent.

    \item \textbf{Theorem 2.7.4}
        All states in a finite closed communicating class are positive recurrent

    \item \textbf{Theorem 2.7.5} All states in an open communicating class are transient.

\end{itemize}

\subsection{Periodicity of chains}

\begin{itemize}

    \item \textbf{Definition 2.8.1} \emph{Period} \\
        The \emph{period} $d$ of a state $i$ is the greatest common factor of
        $\{ n \geq 0 : p_{ii}^{(n)} > 0 \}$.
        If $d=1$, the state is called aperiodic, and for $d \geq 2$, it is called periodic.

    \item \textbf{Theorem 2.8.2} Period is a class property, that is
        $i \leftrightarrow j \implies d_i = d_j$.

\end{itemize}

\subsection{Stationary probabilities: Aperiodic case}

\begin{itemize}

    \item \textbf{Definition 2.9.1} \emph{Stationary distribution} \\
        A distribution $\pi = (\pi_1, \pi_2, \ldots) \geq 0$ is \emph{stationary} if it satisfies the \emph{global balance equations}, i.e.
        \[
            \pi = \pi P,\ \text{ and } \sum_{j \in S} \pi_j = 1
        \]
        (i.e.\ the next state distribution is the same as the current distribution.)\\
        A chain with a stationary distribution is said to be in a stationary state
        or steady state.

    \item \textbf{Proposition 2.9.2} \\
        If a chain is initially in a stationary distribution, $a^{(0)} = \pi$,
        then $a^{(n)} = \pi \ \forall n \geq 0$.

    \item Intuition
        \[
            \lim_{n \to \infty} p_{jj}^{(n)} = p_{jj}^* = \frac{1}{m_{jj}} > 0
        \]
        Think about our simple weather example and ask for the probability than in a million
        years it will rain if it rains today.
        Now what is this probability a million years and one day later?
        Intuitively, we can argue that these probabilities should be equal
        after a really long time,
        and it should be equal to the long run average proportions of the days that are rainy.

    \item \textbf{Theorem 2.9.3}
        \begin{enumerate}
            \item For aperiodic, irreducible chains, the \emph{limiting probabilities}
                are independent of the initial state, that is $\forall i, j, \in S$,
                \[
                    \lim_{n \to \infty}p_{jj}^{(n)} =
                    \lim_{n \to \infty} p_{ij}^{(n)} \equiv \pi_j
                \]
                and we call this limit $\pi_j$.
            \item If the chain is also positive recurrent then the limiting probability
                distribution is the unique stationary distribution.
        \end{enumerate}

\end{itemize}

\subsection{Stationary probabilities: Periodic case}

\begin{itemize}

    \item \textbf{Theorem 2.10.1} (analogous to Theorem 2.9.3)
        \begin{enumerate}
            \item For irreducible chains, the steady-state probabilities $\pi_j$
                are independent of the initial state, that is,
                \[
                    \pi_j = p_{jj}^* = p_{ij}^*
                \]
                for every $i, j \in S$.
            \item If the chain is also positive recurrent then the limiting probability
                distribution $\pi$ is the unique stationary distribution
        \end{enumerate}

    \item Summary \\
        \begin{tabular}{c c c}
            \toprule
            & \emph{Aperiodic} & \emph{Periodic} \\
            \midrule
            $\lim_{n \to \infty}p_{ij}^{(n)}$ & Exist & Do not exist \\
            % \midrule
            $p_{ij}^*$ & Exist & Exist \\
            % \midrule
            Interpretation for $\pi_j$ & Stationary probability, Limiting probability &
            Stationary probability \\
            \bottomrule
        \end{tabular} \\
        \begin{align*}{}
            {\pi_j} & \equiv {p_{jj}^*} = {p_{ij}^*} = {\lim_{n \to \infty} p_{jj}^{(n)}} =
            {\lim_{n \to \infty} p_{ij}^{(n)}} & \text{(aperiodic)} \\
            {\pi_j} & \equiv {p_{jj}^*} = {p_{ij}^*} & \text{(periodic)}
        \end{align*}

\end{itemize}

\subsection{First passage probabilities and times}

\begin{itemize}

    \item Preamble \\
        \[
            T_j = \text{min}\{n \geq 1 : X_n = j\}
        \]
        Define the \emph{first passage time} of a chain to a set $A \subset S$ as
        \[
            \widehat{T}_A = \text{min} \{n \geq 0 : X_n \in A \}
        \]
        Note that $\widehat{T}_A = 0$ for $X_0 \in A$,
        that is if we are already in $A$, it takes no time to get there.

        Otherwise, the \emph{first passage time} $\widehat{T}_A$ is identical to the
        \emph{first arrival time}
        \[
            T_A = \text{min} \{n \geq 1 : X_n \in A\}
        \]
        for $X_0 \in S - A$.
        We defined this variation for convenience.
    \item \textbf{Theorem 2.11.1} \\
        Let $A, B \subset S$, with
        $P(\text{min}\{\widehat{T}_A,\widehat{T}_B\} < \infty | X_0 = i) = 1, \ \forall i$. \\
        Then then probability
        $h_i \equiv P(\widehat{T}_A < \widehat{T}_B | X_0 = i)$
        of reaching set $A$ before set $B$ when starting from state $i$ satisfies
        \[
            h_i \equiv P(\widehat{T}_A < \widehat{T}_B | X_0 = i) =
            \begin{cases}{}
                0 & \text{if} \ i \in B \\
                1 & \text{if} \ i \in A \\
                \sum_{j \in S} P_{ij}h_j & \text{if} \ i \in S - (A \cup B) \\
            \end{cases}
        \]
    \item \textbf{Theorem 2.11.2} \\
        Let $A \subset S$, with $P(\widehat{T}_A < \infty | X_0 = i) = 1, \ \forall i$.
        Then the mean time $g_i \equiv E(\widehat{T}_A | X_0 = i)$
        to reach set $A$ when starting from state $i$ satisfies
        \[
            g_i \equiv E(\widehat{T}_A | X_0 = i) =
            \begin{cases}{}
                0 & \text{if} \ i \in A \\
                1 + \sum_{j \in S} P_{ij}g_j & \text{if} \ i \in S - A \\
            \end{cases}
        \]
\end{itemize}

\subsection{Costs and rewards}

\begin{itemize}
    \item Preamble \\
        At step $n$ we incur a cost of $c(X_n)$
\end{itemize}

\subsubsection{Long-run average cost}

We can write the \emph{long-run average cost} starting from state $i$ to be
\[
    \psi_i = \lim_{N \to \infty} \frac{1}{N+1} E\left( \sum_{n=0}^N c(X_n) | X_0 = i \right)
\]
The long-run average cost is independent of the initial state and is calculated using steady-state probabilities.
Note that you'll get the same expression for the \emph{long-run mean cost}
\[
    \lim_{n \to \infty} E(c(X_n)|X_0=i) = \sum_{j \in S}c(i)\pi_j
\]
for chains which are also aperiodic.

\subsubsection{Cost in transient states}
Another situation is when only transient states have non-zero costs.
In this case the total final cost can be defined $\sum_{n=0}^\infty c(X_n)$.

\subsection{Reversibility}

\begin{itemize}
    \item \textbf{Definition 2.13.1} \emph{Reversed process} \\
        If ${(X_n)}_{n \in N}$ is a stationary DTMC and we fix an $m$, then the process
        ${(\widetilde{X}_n)}_{0 \leq n \leq m}$ where $\widetilde{X}_n = X_{m-n}$
        is called the reversed process of $X$.
    \item \textbf{Theorem 2.13.2} \\
        The reversed process ${(\widetilde{X}_n)}_{0 \leq n \leq m}$
        is a DTMC with transition probabilities
        \[
            \widetilde{p}_{ij} = \frac{ {\pi_j} {p_{ji}} }{\pi_i}
        \]
    \item \textbf{Definition 2.13.3} \emph{Reversibility} \\
        A stationary DTMC is said to be reversible if the reversed process is stochastically the
        same as the original process,
        that is $\widetilde{p}_{ij} = p_{ij}$, which implies
        \[
            \pi_i \ p_{ij} = \pi_j \ p_{ji}, \ \forall i, j \in S
        \]
        This equation together with $\sum_{i \in S} \pi_i = 1$
        are called the \emph{detailed balance equations}.
    \item \textbf{Corollary 2.13.4} \\
        If $\pi$ satisfies the \emph{detailed balance equations}, then it also satisfies the
        \emph{global balance equations}.
    \item A \emph{tree DTMC} has the following properties:
        \begin{itemize}
            \item $p_{ij} > 0 \implies p_{ji} > 0$
            \item No cycles in its state diagram
        \end{itemize}
    \item \textbf{Theorem 2.13.7} A stationary \emph{tree DTMC} is reversible.
    \item \textbf{Corollary 2.13.8} \\
        A \emph{stationary random walk} that is a positive recurrent random walk in stationarity,
        is reversible.
\end{itemize}

\section{Poisson Processes}

\subsection{Exponential Random Variable}

\begin{itemize}
    \item \textbf{Definition 3.1.1:} \emph{Exponential random variable} \\
        A continuous non-negative random variable $X$ is called exponential with rate $\lambda$
        if its cumulative distribution function is
        \[
            P(X \leq x) = F(x) = 1 - e ^{-\lambda x}
        \]
        Consequently, its density is
        \[
            f(x) = \lambda e^{-\lambda x}
        \]
        both for $x \geq 0$, and zero otherwise.
    \item \textbf{Theorem 3.1.2} \\
        The $r$-th moment of the exponential random variable with rate $\lambda$ is given by
        \[
            E(X^r) = \frac{r!}{\lambda^r}
        \]
\end{itemize}

\subsubsection{Memoryless property}
The \emph{memoryless property} of a stochastic distribution can be stated as:
\[
    P(X>s+t|X>s) = \frac{P(X>s + t, X>s)}{P(X>s)} = \frac{e^{-\lambda (s+t)}}{e^{-\lambda s}}
    = e^{-\lambda t} = P(X>t)
\]
\begin{itemize}

    \item \textbf{Theorem 3.1.3} \\
        The only continuous distribution which has a support $[{0,\infty}]$
        % Change above to a open interval
        with memoryless property is the exponential.

\end{itemize}

\subsubsection{Properties of minimum of two exponentials}
\begin{align*}{}
    P(X_1 < X_2) &= \frac{\lambda_1}{\lambda_1 + \lambda_2} \\
    Z = \min(X_1, X_2) \implies
    P(Z>x) &= \mathrm{e}^{-(\lambda_1 + \lambda_2)x} \\
    P(Z > x, X_1 < X_2) &= P(Z>x) P(X_1 < X_2) \quad \text{i.e.\ they are independent}
\end{align*}

\subsubsection{Strong memoryless property}
\begin{itemize}
    \item \textbf{Theorem 3.14} \\
        If $X_2$ is an exponential random variable with rate $\lambda$ and $X_1$
        is an independent non-negative continuous random variable, then $\forall x \geq 0$
        \[
            P(X_2 > X_1 _ x | X_2 > X_1) = P(X_2 > x) = e^{-\lambda x}
        \]
\end{itemize}

\subsubsection{Sums of I.I.D exponentials}

\begin{itemize}
    \item \textbf{Theorem 3.1.5} \\
        If $Z = X_1 + X_2 + \cdots + X_n$, where $X_i \approx \text{exp}(\lambda)$ for all $i$
        and independent, then $Z$ is called the gamma $(n, \lambda)$ random variable and its
        density function is given by
        \[
            f_n(z) = \lambda e^{-\lamda x} \frac{{(\lambda z)}^{n-1}}{(n-1)!}
        \]
\end{itemize}

\subsection{Poisson Processes}

\begin{itemize}
    \item \textbf{Definition 3.2.1} \\
        Let $\tau_i$ be independent exponential $(\lambda)$ random variables,
        $S_0 = 0, s_n = \tau_1 + \tau_2 + \cdots + \tau_n$ and
        $N_t = \text{max}\{n \geq 0 : S_n \leq t \}$.

        Then ${(N_t)}_{t \in \mathbb{R}_{\geq 0}}$ is a \emph{Poisson process} with rate parameter
        $\lambda$, or briefly PP$(\lambda)$.
    \item \textbf{Theorem 3.2.2} \\
        If ${(N-t)}_{t \geq 0}$ is a $\mathrm{PP}(\lambda)$,
        then $N_t$ follows a Poisson distribution with rate $\lambda t$ for any $t$,
        that is
        \[
            P(N_t = k) = \frac{\mathrm{e}^{-\lambda t} {(\lambda t)}^k}{k!}
        \]
    \item \textbf{Definition}
        \[
            {\left(N_t^{(s)}\right)}_{t\geq 0} = N_{s+t} - N_s \quad ; \quad \forall t \geq 0
        \]
        I.e.\ \emph{resetting} the counter at $s$.
    \item \textbf{Theorem 3.2.4}
        The process ${(N_t^{(s)})}_{t \geq 0}$ is a $\mathrm{PP}(\lambda)$,
        and it is independent of ${(N_u)}_{0 \leq u \leq s}$.
    \item \textbf{Definition 3.2.6}
        \begin{enumerate}
            \item A process ${(N_t)}_{t\geq 0}$ is said to have \emph{stationary increments}
                if $N_{s+t} - N_s$ is identically distributed for all $s$,
                that is the distribution does not depend on $s$.
            \item A process ${(N_t)}_{t\geq 0}$ is said to have independent increments if the
                increments of the distribution is independent for non-overlapping intervals,
                that is $N_{s_1 + t_1} - N_{s_1}$ and $N_{s_2 + t_2} - N_{s_2}$
                are independent if $[s_1,s_2 + t_1] \cap [s_2, s_2 + t_2] = \emptyset$.
        \end{enumerate}
    \item \textbf{Theorem 3.2.7}
        ${(N_t)}_{t \geq 0}$ is a $\mathrm{PP}(\lambda)$ if and only if
        \begin{enumerate}
            \item it has stationary and independent increments
            \item $N_t$ is a Poisson $(\lamda t)$ random variable for all $t$.
        \end{enumerate}
    \item \textbf{Definition 3.2.8}
        A function $f(x)$ is said to be a $o(x)$ function, if
        $\lim_{x \to 0} \frac{f(x)}{x} = 0$.
    \item \textbf{Theorem 3.2.10}
        ${(N_t)}_{t \geq 0}$ is a $\mathrm{PP}(\lambda)$ if and only if
        \begin{enumerate}
            \item it has stationary and independent increments
            \item
                \begin{align*}{}
                    P(N_h = 0)    & = 1 - \lambda h + o(h) \\
                    P(N_h = 1)    & = \lambda h + o(h) \\
                    P(N_h \geq 2) & = o(h)
                \end{align*}
        \end{enumerate}
\end{itemize}

\subsection{Super-positioning and Splitting}

\begin{itemize}
    \item \textbf{Theorem 3.3.1}
        Let ${(N_t^{(i)})}_{t\geq 0}$ be $\mathrm{PP}(\lambda _I)$ for each $i = 1, \ldots, k$.
        Then, with $N_t = N_t^{(1)} + \cdots + N_t^{(k)}$, ${(N_t)}_{t\geq 0}$ is a
        $\mathrm{PP}(\lambda_1 + \cdots + \lambda_k)$.

    \item \textbf{Theorem 3.3.2}
        Suppose ${(N_t)}_m{t \geq 0}$ is a $\mathrm{PP}(\lambda)$.
        If an event in this process is of type $i$ with probability $p_i$
        independent of other events, where $\sum_{i=1}^k p_i = 1$, then the processes

        ${(N_t^{(1)})}_{t\geq 0}, {(N_t^{(2)})}_{t \geq 0}, \ldots, {(N_t^{(k)})}_{t \geq 0}$
        are independent Poisson processes with rates $\lambda p_1, \lambda p_2, \ldots, \lambda p_k$
        respectively.

\end{itemize}

\subsection{Campbell's Theorem: Uniform Order Statistics}

\begin{itemize}
    \item \textbf{Theorem 3.4.1 Campbell's Theorem} \\
        Let $S_n$ be the event times for a Poisson process.

        If $N_t = n$ is given, then the vector $(S_1, S_2, \ldots, \S_n)$
        follows the distribution of ordered independent uniform variables
        $(U_{(1)}, U_{(2)}, \ldots, U_{(n)})$.

        Consequently, the unordered set of arrival times
        $\{S_1, \ldots, S_n\}$ has the same distribution as $\{U_1, \ldots, U_n\}$.

        In other words, $\{S_1, \ldots, S_n\}$ is uniformly distributed.
\end{itemize}

\subsection{Non-homogeneous Poisson Process}

Having now relaxed the stationary assumption, $\lambda(t)$ depends on time.

\begin{itemize}
    \item \textbf{Definition 3.5.1}
        A counting process ${(N_t)}_{t\geq 0}$ is a non-homogeneous Poisson process if
        \begin{enumerate}
            \item it has independent increments
            \item
                \begin{align*}{}
                    P(N_{t+h} - N_t = 0)    & = 1 - \lambda(t)h + o(h) \\
                    P(N_{t+h} - N_t = 1)    & = \lambda(t)h + o(h) \\
                    P(N_{t+h} - N_t \geq 2) & = o(h)
                \end{align*}
                Interestingly, the number of arrivals at any fixed time $t$ is still Poisson.
        \end{enumerate}
    \item \textbf{Theorem 3.5.2}
        Define
        \[
            \Lambda(t) = \int_0^t \lambda(u) \mathrm{d}u
        \]
        Then $N_t$ is a Poisson $\Lambda(t)$ random variable for any $t$.
    \item \textbf{Corollary 3.5.3}
        $N_t - N_s$ is a Poisson random variable with parameter
        $\Lambda(t) - \Lambda(s) = \int_s^t \lambda(u) \mathrm{d}u$.
    \item \textbf{Corollary 3.5.5}
        The super-positioning and splitting properties are true for non-homogeneous
        Poisson processes as well.

\end{itemize}

\subsubsection{Event Times for Non-homogeneous PP}
\begin{itemize}
    \item \textbf{Theorem 3.5.6} \\
        Let ${(N_t)}_{t \geq 0}$ be a non-homogeneous PP, then if $N_t = k$ is given, then
        \[
            (S_1, S_2, \ldots, S_k) \sim (U_{(1)}, U_{(2)}, \ldots, U_{(k)})
        \]
\end{itemize}

\subsection{Compound Poisson Processes}

\section{Continuous Time Markov Chains}
\subsection{Basic Definitions}

\begin{itemize}
    \item \textbf{Notation} \\
        $\tau_i$ represents the time between states $\widetilde{X}_{i-1}$ and $\widetilde{X}_i$.
        \[
            S_n = \sum_{i=1}^n \tau_n
        \]
    \item \textbf{Definition 4.1.1} \emph{Continuous Time Markov Chain} \\
        The continuous time stochastic process $(X_t){t \geq 0}$ is called a
        \emph{continuous time markov chain (CTMC)} if
        \begin{enumerate}
            \item each duration $\tau_n$ is an exponential random variable with rate $q_i > 0$
                which depends only on state $\widetilde{X}_{n-1} = i$  if the process leaves
            \item the corresponding (embedded) discrete time process
                $(\widetilde{X}_{n \in \mathbb{N}})$ is a DTMC with $\widetilde{p}_{ii} = 0$
                for all $i$.
        \end{enumerate}
        More formally
        \begin{align*}{}
            P(\widetilde{X}_n = j, \tau_n > y | \widetilde{X}_{n-1} = i, \tau_{n-1},
            \widetilde{X}_{n-2}, \tau_{n-2}, \ldots, \widetilde{X}_0)
            &= P(\widetilde{X}_n = j, \tau_n > y | \widetilde{X}_{n-1} = i) \\
            &= \widetilde{p}_{ij} e^{-q_i y}
        \end{align*}
    \item \textbf{Definition 4.1.2} \emph{Markov Property} \\
        A continuous time process ${(X_t)}_{t \geq 0}$ has the \emph{Markov property} if
        for any $0 \leq s_0 < s_1 < \cdots < s_n < s$, any $t \geq 0$ and any possible states
        $i_0, \ldots, i_n, i, j$ we have
        \[
            P(X_{s+t} = j | X_s = i, X_{s_n} = i_n, \ldots, X_{s_0} - i_0) =
            P(X_{s+t} = j | X_s = i)
        \]
    \item \textbf{Theorem 4.1.3} The CTMC ${(X_t)}_{t \geq 0}$ has the Markov Property
    \item \textbf{Definition 4.1.7} \emph{Generator} \\
        The \emph{rate matrix} or the \emph{generator} $Q$ of the CTMC
        ${(X_t)}_{t\geq0}$ is defined through its elements
        \begin{align*}{}
            q_{ij} &= q_i \widetilde{p}_{ij}, \quad \text{if} \ i \neq j \\
            q_{ii} &= -q_i = - \sum_{j \in S, j \neq i} q_{ij}
        \end{align*}
        Here $q_{ij}$ is called the jump rate from $i$ to $j$.
\end{itemize}

\subsection{Chapman-Kolmogorov Equations}

\begin{itemize}
    \item \textbf{Notation} \\
        The transition matrix $P(t)$ has entries
        \[
            P_{ij}(t) = P(X_t = j | X_0 = 0)
        \]
        The initial distribution is defined to be
        \[
            a_i^{(0)} = P(X_0 = i)
        \]
    \item \textbf{Theorem 4.2.1} \\
        The matrix $P(t)$ and the initial distribution $a^{(0)}$ completely characterises the CTMC,
        that is the probability
        \[
            P(X_{t_1} = i_1, X_{t_2} = i_2, \ldots, X_{t_k} = i_k)
        \]
        can be asserted by only knowing $P(t)$ and $a^{(0)}$.
    \item \textbf{Theorem 4.2.2} \\
        $P(t)$ has the following properties
        \begin{enumerate}
            \item $p_{ij}(t) \geq 0$
            \item $\sum_{j \in S} p_{ij}(t) = 1$ for all $t \geq 0$
            \item \emph{Chapman-Kolmogorov Equations} \\
                $P(t+s) = P(t) P(s)$, that is $p_{ij}(t+s) = \sum_{k \in S} p_{ik}(t) p_{kj}(s)$
        \end{enumerate}
    \item \textbf{Theorem 4.2.3} \\
        $P'(0) = Q$, that is
        \[
            p'_{ii}(0) = -q_i = q_{ii}, \quad p'_{ij}(0) = q_{ij} \quad \text{for} \ i \neq j
        \]
\end{itemize}

\subsection{Backward and Forward Equations}

\begin{itemize}
    \item \textbf{Theorem 4.3.1} \\
        Let $P(t)$ be the transition matrix and $Q$ be the generator of a CTMC\@.
        Then $P(t)$ is the unique solution of both the forward Kolmogorov equation
        \[
            P'(t) = P(t) Q \quad \text{that is} \quad p'_{ij}(t) = \sum_{k \in S}p_{ik}(t)q_{kj}
        \]
        and the backward Kolmogorov equation
        \[
            P'(t) = Q P(t) \quad \text{that is} \quad p'_{ij}(t) = \sum_{k \in S}q_{kj}p_{ik}(t)
        \]
        Both backward and forward equations have the initial conditions
        \[
            P(0) = \mathds{1} \quad \text{that is} \quad p_{ii}(0) = 1, p_{ij}(0) = 0 \
            \text{for} \ j \neq i
        \]
    \item \textbf{Theorem 4.3.4} \\
        For finite state spaces the solution of both backward and
        forward equations is
        $P(t) = \mathrm{e}^{Qt}$.

\end{itemize}

\subsection{Transience and Recurrence}

\begin{itemize}
    \item \textbf{Definition 4.4.2} \\
        Where
        \[
            \varrho_{ij} = P(T_j < \infty | X_0 = i), \quad m_{ij} = E(T_j | X_0 = i)
        \]
        We have that state $i$ is
        \begin{enumerate}
            \item transient if $\varrho_{ii} < 1$,
            \item null current if $\varrho = 1$ and $m_{ii} = \infty$,
            \item positive recurrent if $\varrho = 1$ and $m_{ii} < \infty$.
        \end{enumerate}
    \item \textbf{Theorem 4.4.3} \\
        A state $i$ for a CTMC is recurrent if and only if the embedded
        DTMC is recurrent.
    \item \textbf{Theorem 4.4.4} \\
        Let ${(X_t)}_{t \geq 0}$ be an irreducible CTMC\@.
        Suppose $\widetilde{\pi}$ is a positive solution of
        $\widetilde{\pi} = \widetilde{\pi} \widetilde{P}$ where $\widetilde{P}$ is the transition
        matrix of the embedded DTMC\@.
        Then the CTMC is positive recurrent if and only if
        \[
            \sum_{i \in S} \frac{\widetilde{\pi}_i}{q_i} < \infty
        \]
\end{itemize}

\subsection{Stationary probabilities}

\begin{itemize}
    \item \textbf{Theorem 4.5.1} \\
        Let ${(X_t)}_{t \geq 0}$ be an irreducible CTMC with limiting distribution $\pi_i$.
        The limiting distribution is the unique stationary distribution,
        that is the unique solution of the global balance equations
        \[
            \pi Q = 0, \quad \sum_{i \in S} \pi_i = 1
        \]
        if and only if the CTMC is positive recurrent.
    \item \textbf{Theorem 4.5.4} \\
        If the detailed balance equations are satisfied, that is
        \[
            \pi_i q_{ij} = \pi_j q_{ji} \quad \forall i \neq j, \quad \text{and} \
            \sum_i \pi_i = 1
        \]
        then also the global balance equations are satisfied, so $\pi$ is the
        stationary distribution.
\end{itemize}

\subsection{First Passage Properties}
\begin{itemize}
    \item \textbf{Definition} \emph{First passage time} \\
        \[
            \widehat{T}_A = \min \{t \geq 0 : X_t \in A \}
        \]
    \item \textbf{Theorem 4.6.1} \\
        Let $A, B \subset S$, such that
        $P(\min \{\widehat{T}_A, \widehat{T}_B\} < \infty | X_0 = i) = 1, \ \forall i$. \\
        Then the probability $h_i = P(\widehat{T}_A < \widehat{T}_B | X_0 = i)$
        of reaching set $A$ before set $B$ when starting from state $i$ satisfies
        \begin{center}{}
            \begin{tabular}{l l}
                $h_i = 0$                       & if $i \in B$ \\
                $h_i = 1$                       & if $i \in A$ \\
                $\sum_{j \in S} q_{ij} h_j = 0$ & if $i \in S - (A \cup B)$
            \end{tabular}
        \end{center}
    \item \textbf{Theorem 4.6.2} \\
        Let $A \subset S$, such that $P(\widehat{T}_A < \infty | X_0 = i) = 1, \ \forall i$. \\
        Then the mean time $g_i = E(\widehat{T}_A | X_0 = i)$ to reach $A$ when starting from state
        $i$ satisfies

        \begin{center}{}
            \begin{tabular}{l l}
                $g_i = 0$                        & if $i \in A$ \\
                $\sum_{j \in S} q_{ij} g_j = -1$ & if $i \in S -S A$
            \end{tabular}
        \end{center}
\end{itemize}

\subsection{Costs and Rewards}

\begin{itemize}
    \item \textbf{Definition} \emph{Long run average cost per unit time} \\
        Where $c(i)$ is the \emph{cost per unit time}, we define
        \[
            \psi_i = \lim_{T \to \infty} \frac{1}{T} E
            \left(\int_0^T c(X_t) \mathrm{d}t | X_0 = i\right)
        \]
    \item \textbf{Theorem 4.7.1}
        For an irreducible positive recurrent chain, we have
        \[
            \psi_i = \sum_{j \in S} \pi_j c(j).
        \]
\end{itemize}

\end{document}
